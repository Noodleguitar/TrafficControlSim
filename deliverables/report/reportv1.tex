\documentclass{article}
\usepackage{bnaic}


%% if your are not using LaTeX2e use instead
%% \documentstyle[bnaic]{article}

%% begin document with title, author and affiliations

\title{\textbf{\huge Traffic Control Simulation - Report v.1}}
\author{Elbert Fliek (s1917188) \affila \and
    Albert Thie (s1652184) \affila \and
    No\"el L\"uneburg (s1773135) \affila}
\date{\affila\ \textit{University of Groningen \\
\today}}

\pagestyle{empty}

\begin{document}
\ttl
\thispagestyle{empty}

\begin{abstract}
    Improving the amount of thoroughfare at crossroads using traffic lights has been a topic of research in urban planning and traffic engineering. Multiple systems have been designed to increase traffic flow in intersections using sensors in traffic lights and optimization based on traffic data. Our research shows the advantage of applying a state of art sorting algorithm and combining this with information about the traveling direction of individual cars. Thus taking advantage of communicating traffic lights and individual cars, we hope to improve traffic flow in intersections. Results to be added in a later version.
% Simulations show..."enter results"  
\end{abstract}

%\begin{abstract}
%\noindent
%\end{abstract}


\section{Introduction}
\subsection{The problem}
Optimizing the traffic flow is a problem which has been studied with computational models in a variety of ways. One possible approach is modelling the behavior of traffic by using natural laws, such as those governing the expansions and transfer of kinetic gas \cite{helbing2001master}. This however, does not solve the problem of optimizing the traffic lights at an intersection. To increase the amount of traffic that can pass an intersection, and reduce the wait time at the intersection, other models are needed.

\subsection{State of the Art}
According to Li et. al\cite{survey2014} the problem of intersection management can be tackled with a combination of not only optimizing the sorting algorithm that schedules green lights based on queue length, but also using predictions of future traffic reaching the intersection. These predictions can be based on data gathered over time at an intersection, thus identifying times of heavy traffic. Their overview however only identifies possible approaches to the problem and not an actual implementation that can be simulated.

L{\"a}mmer et. al. \cite{self_control2008} however created a simulation based on fluid dynamics.  L{\"a}mmer et. al. view the problem of intersections as a multi-agent problem, where both traffic lights and individual vehicles interact together. Traffic lights assign priority among each other by detecting not only waiting traffic, but also oncoming traffic. Oncoming traffic indicates its current speed and project trajectory, allowing traffic lights to take this into account when deciding priority among themselves. This allows for platoon of oncoming vehicles to influence decision making in a traffic light. Through the pressure of "future"  cars, the green wave effect for a platoon of cars can be created emergently.

This approach was tested not only in traffic simulations, but also implemented in two real life intersections in the city of Dresden \cite{site}. The real life application notably increased traffic flow and allowed for preferential treatment for public transport vehicles, further helping the city's infrastructure.


\subsection{New Idea}
The current simulation by L{\"a}mmer et. al. assumes that while the problem is modelled as a multi-agent system, the traffic lights themselves do all of the decision making. The only "input" from vehicles is their presence and probable trajectory and speed, which are detected by cameras. However, with the development of smart cars, vehicles could also transmit the information about their destination to the traffic lights. More importantly, they can transmit in which direction they will go once on the intersection. This information allows traffic lights to increase their efficiency by taking this route information into account. If for example two opposing lanes only contain vehicles which wish to go straight on or move to their right (assuming a country where the driving side is on the right), the lights of both sides could remain green until a vehicle appears that wishes to go left. This is useful in intersections where dedicated lanes are not an option due to space or financial limitations. 



\section{Method}
The dynamic queue by L{\"a}mmer et. al. is compared to regular traffic control systems in a simulation environment. Both systems use a four square intersections with vehicles approaching at random intervals from all sides. Intervals are increased and decreased to measure the effects of high and low traffic flow situations. 

Time in the simulation is measured in frames, while individual of vehicles is measured in traversed pixels. This abstraction allows for vehicles speed, brake and acceleration time to be measured in screen movement.

The regular traffic system used a fixed green time for each traffic light, followed by a period of 120 frames of yellow light. After this period a new traffic light was randomly set to green.

The dynamic system uses a queue based approach, where the amount of vehicles in the queue determines not only which light will be green next, but also the time the light will remain green. Green time is decided by the amount of present and approaching cars in the queue, the amount of start up time needed for cars to start moving and the time needed to clear the intersection after all cars have left. 

%Pressure is exerted 


\subsection{Simulation environment}
% TODO Screenshot(s) of simulation
We have developed an environment that allows simulation of traffic on intersections. Traffic light status and vehicle activity is visualized from a top-down view.

A simulated traffic scenario contains an intersection with lanes going towards and away from it. A variety of vehicles occupy these lanes. This section describes properties that can be adjusted for each entity.

\paragraph{Intersection} ~\\
An intersection within our simulation can connect roads in up to four directions; north, south, east and west. \\
\textbf{Number of lanes} ~ The number of lanes in each direction is used to modify traffic capacity and affect the complexity of the traffic situation.\\
\textbf{Traffic lights} ~ Traffic lights can be placed on lanes entering the intersection, signalling whether or not approaching vehicles can proceed.

\paragraph{Vehicle} ~\\
Each vehicle is tied to a lane heading toward or away from an intersection. Vehicles are individual agents with the following properties:\\
\textbf{Length} ~ Length of the vehicle.\\
\textbf{Turning rate} ~ Duration for this vehicle to fully turn either right or left on an intersection.\\
\textbf{Deceleration} ~ Braking constant, determines the rate at which the speed diminishes when a vehicle is slowing down.\\
\textbf{Acceleration} ~ Acceleration constant, determines the rate at which the speed increases when a vehicle accelerates.\\
\textbf{Maximum speed} ~ Maximum speed of the vehicle.\\

\subsection{Experimental setup}
\label{exp_setup}
Two traffic control systems are compared. The first is a regular system which operates based on inputs from road sensors in front of traffic lights. Each vehicle that passes one of these sensors generates a pulse. The total number of pulses determines the priority of the lane going into the intersection. The lane with the highest current priority will show a green signal, but only after every other lane has shown a red signal. This entails that only one lane can be active at one time. To avoid rapid light changes, a minimum green signal duration is enforced.

The second system is similar in that it operates using a lane priority system. However, more information is available, such as the type of each vehicle, and traffic density in a particular direction. Each traffic signal has access to traffic information of the lane that it is assigned to. Communication between traffic signals of different lanes determines which lane is allowed access to the intersection at any point in time.

To quantify the performance of both systems we make use of two measures: the average waiting time of all vehicles, and the average time it takes for each vehicle to completely follow its path from entering the simulated world until exiting.

\section{Results}

\section{Conclusion}


\bibliographystyle{plain}
\bibliography{literature}



\end{document}



